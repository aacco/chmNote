\documentclass[dvipdfmx,uplatex]{jsarticle}
\usepackage{../sty/macros}
\usepackage{../sty/phe}
\usepackage[version=3]{mhchem}
\usepackage{url}
 
\begin{document}
\subsection{検出反応}
\begin{table}[htb]
\caption{代表的な検出反応}
	\begin{tabular}{|l|l|l|}
	反応名や検出剤 & 反応物 & 呈色,反応 \\ \hline \hline
	ビウレット反応 & タンパク質 & 赤紫\footnotemark \\
	キサントプロテイン反応&ベンゼン環を含むタンパク質&黄色沈殿$\xrightarrow{塩基性}$橙黄色\footnotemark \\
	ニンヒドリン反応&アミノ酸(濃度で濃淡変化)&青紫,赤紫\footnotemark \\
	硫黄反応 &\ce{S}を含むタンパク質&黒色沈殿;硫化鉛\ce{PbS}\footnotemark \\
	フェーリング反応&還元性物質\footnotemark&赤褐色沈殿;酸化銅(I)\ce{Cu2O}\footnotemark \\
	銀鏡反応& & 黒色沈殿\ce{Ag}\footnotemark \\
	シッフ試薬(フクシンアルデヒド試薬)&アルデヒド(ケトンは不活性)&赤,紫\footnotemark \\
	臭素水&二重,三重結合&脱色 \\
	ナトリウム&アルコール,フェノール&(アルコキシド,\ce{H2}の発生) \\
	カリウム& & \\
	二クロム酸カリウム\ce{K2Cr2O7}&第一,二級アルコール&(アルデヒド,ケトンの生成) \\
	 &アニリン\ce{{\phenyl} NH2}&黒色沈殿;アニリンブラック \\
	塩化鉄(III):\ce{FeCl3}&フェノール類\ce{{\phenyl} OH}&赤紫,青紫\footnotemark \\
	さらし粉:\ce{CaCl(ClO).H2O}&アニリン\ce{{\phenyl} NH2}&赤紫\footnotemark \\
	ヨウ素デンプン反応(\ce{KI + I2})&デンプン&青紫 \\
	ヨウ化カリウムデンプン紙&酸化作用物質&青変\footnotemark \\
	褐輪反応&\ce{NO3- , NO2-}&褐色環\footnotemark \\
	ヨードホルム反応&アセチル基:\ce{CH3CO - R}&黄色\footnotemark \\ \hline
	\end{tabular}
\end{table}

\addtocounter{footnote}{-13} % 表中で使った脚注の数だけfootnoteカウンタから引く
\stepcounter{footnote}\footnotetext{強塩基と\ce{CuSO4}で連続2つ以上のペプチド結合が銅イオン\ce{Cu^2+}とキレートを作る}
\stepcounter{footnote}\footnotetext
{
	\ce{HNO3}.
	TyrやPheのベンゼン環がニトロ化し黄色.
	とりわけフェノール性ヒドロキシ基のo,p-配向性でTyrと起こりやすく,
	塩基性にすると\ce{-OH}から\ce{H+}が脱離,キノン構造ができ発色が濃くなる.
}
\stepcounter{footnote}\footnotetext{ニンヒドリンと$\alpha$-アミノ酸からルーヘマン紫が生じる}
\stepcounter{footnote}\footnotetext{\ce{NaOH(s)}を加えて加熱,中和後酢酸鉛(II)aq:\ce{(CH3COO)2Pb}を加える}
\stepcounter{footnote}\footnotetext{アルデヒドや糖類など.}
\stepcounter{footnote}\footnotetext
{
	\ce{CuSO4 + $ロッシェル塩$ + NaOH + R - CHO -> Cu2O} \\
	Fehling液はA液:\ce{CuSO4aq}と
	B液:\ce{KOOCCH(OH)CH(OH)COONa + NaOH}を直前に混合して生成.
	\ce{Cu2+}を酒石酸イオンのキレート錯体でアルカリから保護している.
}
\stepcounter{footnote}\footnotetext{\ce{[Ag(NH3)2]+ + e- -> Ag + 2NH3}(還元反応)}
\stepcounter{footnote}\footnotetext{マゼンタaqに\ce{SO2}を作用させて生成.ケトンと反応しにくいのは周囲の基からの電子給与や立体障害による.}
\stepcounter{footnote}\footnotetext{\ce{Fe^3+}に\ce{{\phenyl} O-}が配位.\ce{O}から\ce{Fe^3+}の3d軌道へ電子が遷移することによる呈色.}
\stepcounter{footnote}\footnotetext{強い酸化剤である\ce{ClO-}により酸化される.}
\stepcounter{footnote}\footnotetext{\ce{2I- -> I2 + 2e-}と酸化され,ヨウ素デンプン反応を起こす.}
\stepcounter{footnote}\footnotetext
{
	\ce{NO3-}と\ce{FeSO4}を含む水溶液に濃硫酸を注ぐと濃硫酸が沈み込み,
	境界面で\ce{NO3-}が\ce{Fe^2+}を酸化し,\ce{NO}を生じる.これが配位し,
	\ce{[Fe(NO)(H2O)5]^2+}(褐色)が生成して呈色.
}
\stepcounter{footnote}\footnotetext
{
	ハロホルム反応の一種.一般にはアセチル基を持つ有機化合物に
	ハロゲン化剤と塩基を作用させると,トリハロメタン(ハロホルム)が得られる: \\
	\ce{CH3CO - R $\xrightarrow[Base]{{\rm X_2}}$ R - COO- + CHX3}
}


\newpage

\begin{table}
\caption{検出反応2}
	\begin{tabular}{|l|l|l|}
	シュバイツァー試薬\ce{[Cu(NH3)4]^2+}&セルロース\ce{(C6H10O5)_n}&銅アンモニアレーヨン \\
	ネスラー試薬\ce{HgI2, KI}&アンモニア&褐色沈殿\ce{NHg2I} \\
	無水硫酸銅&水&青色結晶\ce{CuSO4.5H2O}
	\end{tabular}
\end{table}

\newpage
\url{https://ja.wikipedia.org/wiki/%E3%82%A2%E3%83%AB%E3%83%89%E3%83%BC%E3%83%AB%E5%8F%8D%E5%BF%9C} \\
\url{https://labchem-wako.fujifilm.com/jp/category/synthesis/organic_synthesis/name_reaction/index.html} \\
\small
A \\
Appel Reaction	アッペル反応
	第1・2級アルコールのハロゲン化反応 \\
B \\
Baeyer-Villiger Oxidation	バイヤー・ビリガー酸化反応
	過酸を用いてケトンを酸化しエステルに変換する反応 \\
Beckmann Rearrangement	ベックマン転位反応
	ケトンから作られたオキシムを経由するN-置換アミド合成反応 \\
Birch Reduction	バーチ還元
	芳香族化合物の1,4-ジエンへの還元反応 \\
Borch Reductive Amination	ボーチ還元的アミノ化反応
	NaBH3CNを用いてアルデヒド/ケトンをアミンへ変換する反応 \\
Brown Hydroboration	ブラウンヒドロホウ素化反応
	ボランがアルケン/アルキンに付加する反応 \\
Buchwald-Hartwig Cross Coupling	バックワルド・ハートウィグ クロスカップリング反応
	芳香族ハロゲン化物とアミン/アルコキシドのパラジウム触媒によるクロスカップリング反応 \\
C \\
Cannizzaro Reaction	カニッツァロ反応
	アルデヒドを用いるアルコールとカルボン酸への不均化反応 \\
Claisen Condensation	クライゼン縮合
	2分子のエステルが塩基の存在下で縮合反応してβ-ケトエステルを生成する反応 \\
Curtius Rearrangement	クルチウス転位反応
	酸アジドを熱分解することでイソシアネートを得られる反応 \\
D \\
Diels-Alder Reaction	ディールス・アルダー反応
	共役ジエンにアルケンが付加して6員環構造を生じる反応、[4+2]環状付加反応 \\
F \\
Friedel-Crafts Acylation	フリーデル・クラフツ アシル化反応
	求電子芳香族置換によるアシル化反応 \\
Friedel-Crafts Alkylation	フリーデル・クラフツ アルキル化反応
	求電子芳香族置換によるアルキル化反応 \\
G \\
Gabriel Amine Synthesis	ガブリエルアミン合成反応
	ハロゲン化アルキルとフタルイミドカリウムから一級アミンを合成する反応 \\
H \\
Horner-Wadsworth-Emmons (HWE) Reaction	ホーナー・ワズワース・エモンス反応
	ホスホン酸ジエステルのイリドがアルデヒドまたはケトンとWittig反応類似の機構を経てα,β-不飽和エステルが得られる反応 \\
J \\
Jones Oxidation	ジョーンズ酸化反応
	酸化クロム(VI)を用いたアルコールの酸化反応 \\
K \\
Knoevenagel Condensation	クネーフェナーゲル縮合反応
	活性メチレン化合物をアルデヒド/ケトンと縮合させてアルケンを得る反応 \\
M \\
Malaprade Glycol Oxidative Cleavage	マラプラード グリコール酸化開裂反応
	過よう素酸もしくは過よう素酸ナトリウムを用いる1,2-ジオール(グリコール)の酸化的開裂反応 \\
Mannich Reaction	マンニッヒ反応
	第二級アミン・アルデヒド・ケトンを酸性条件下で反応させ、β-アミノカルボニル化合物を得る反応 \\ 
Mitsunobu Reaction	光延反応
	一級/二級アルコールをアゾカルボン酸エステルとトリフェニルホスフィンで活性化して行なうSN2反応 \\
Mizoroki-Heck Reaction	溝呂木・ヘック反応
	Pd(0)触媒下で、ハロゲン化アリール/アルケニルを末端オレフィンとクロスカップリングさせ、置換オレフィンを合成する反応 \\
Mukaiyama Aldol Reaction	向山アルドール反応
	ルイス酸触媒を用いたシリルエノールエーテルとカルボニル化合物とのアルドール反応 \\
N \\
Negishi Cross Coupling	根岸クロスカップリング反応
	Pd(0)もしくはNi触媒下で進行する有機亜鉛化合物と有機ハロゲン化物のクロスカップリング反応 \\
P \\
Prilezhaev Epoxidation	プリリツェフ エポキシ化反応
	過酸を用いてオレフィンをエポキシドに変換する反応 \\
R \\
Robinson Annulation	ロビンソン環形成反応
	Wieland-Miescherケトンを合成する方法 \\
S \\
Sandmeyer Reaction	ザンドマイヤー反応
	Cu(I)イオン存在下での芳香族ジアゾ化合物の置換基変換反応 \\
Sharpless-Katsuki Asymmetric Epoxidation	シャープレス・香月不斉エポキシ化反応
	アリルアルコールの不斉エポキシ化 \\
Sonogashira-Hagihara Cross Coupling	薗頭・萩原クロスカップリング反応
	ハロゲン化アリール/ハロアルカンと、末端アルキンのクロスカップリング反応 \\
Swern Oxidation	スワーン酸化反応
	ジメチルスルホキシド(DMSO)と塩化オキサリル系によるアルコールの酸化反応 \\
T \\
Tsuji-Trost Reaction	辻・トロスト反応
	Pd(0)触媒下、π-アリルパラジウム中間体を経由するアリル位置換反応 \\
V \\
Vilsmeier-Haack Reaction	ヴィルスマイヤー・ハック反応
	DMFとオキシ塩化リン(POCl3)を用いたホルミル化反応 \\
W \\
Welliamson ether synthesis	ウィリアムソンエーテル合成
	アルコキシド/フェノラートとハロゲン化アルキルを経由するエーテルの合成反応 \\
Wittig Reaction	ウィッティヒ反応
	リンイリドとカルボニル化合物からアルケンを合成する反応 \\
Wohl-Ziegler Bromination	ウォール・チーグラー臭素化反応
	N-コハク酸イミド(NBS)による臭素化反応 \\
Wolff-Kishner Reduction	ウォルフ・キシュナー還元反応
	アルデヒドやケトンをヒドラジンと反応させ、ヒドラゾン経由によるメチレンへの還元反応\\

\begin{table}
\begin{center}
\caption{代表的な半反応式一覧}
	\begin{tabular}{|l|l|}
物質/化学式(条件)&	半反応式\\ \hline \hline
酸化剤& \\ \hline
ハロゲン & \ce{X2 + 2e- -> 2X-} \\
オゾン 

(酸性)&	
  \ce{O3 + 2H+ + 2e- -> O2 + H2O}\\
過酸化水素 

(酸性)&	
  \ce{H2O2 + 2H+ + 2e- -> 2H2O}\\
過マンガン酸カリウム 

(酸性)	&
  \ce{MnO4^- + 8H+ + 5e- -> Mn^2+ + 4H2O}\\
過マンガン酸カリウム 

(中性・塩基性)	&
  \ce{MnO4^- + 2H2O + 3e- -> MnO2 + 4OH-}\\
濃硝酸 &
  \ce{HNO3 + H+ + e- -> NO2 + H2O}\\
希硝酸 &
  \ce{HNO3 + 3H+ + 3e- -> NO + 2H2O}\\
熱濃硫酸& 
  \ce{H2SO4 + 2H+ + 2e- -> SO2 + 2H2O}\\
二クロム酸カリウム 

(酸性)	&
  \ce{Cr2O7^2- + 14H+ + 6e- -> 2Cr^3+ + 7H2O}\\
二酸化硫黄 &
  \ce{SO2 + 4H+ + 4e- -> S + 2H2O}\\
酸素 &
  \ce{O2 + 4e- -> 2O^2-}\\ \hline \hline
還元剤& \\ \hline 
金属 	&
  \ce{X} \ce{->} $\ce{X}^{n+}$ + n\ce{e-}\\
塩化スズ(II)& 
  \ce{Sn^2+ -> Sn^4+ + 2e-}\\
硫酸鉄(II) &
  \ce{Fe^2+ -> Fe^3+ + e-}\\
硫化水素 &
  \ce{H2S -> S + 2H+ + 2e-}\\
過酸化水素& 
  \ce{H2O2 -> O2 + 2H+ + 2e-}\\
二酸化硫黄 &
  \ce{SO2 + 2H2O -> SO4^2- + 4H+ + 2e-}\\
シュウ酸 &
  \ce{(COOH)2 -> 2CO2 + 2H+ + 2e-}\\
ヨウ化カリウム 	&
  \ce{2I- -> I2 + 2e-}\\
水素 	&
  \ce{H2 -> 2H+ + 2e-}\\
チオ硫酸ナトリウム &
  \ce{2S2O3^2- -> S4O6^2- + 2e-}\\
	\end{tabular}
\end{center}
\end{table}

\end{document}
