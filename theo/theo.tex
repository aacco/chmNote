\documentclass[dvipdfmx,uplatex]{jsarticle}
\usepackage{macros}
\usepackage[version=3]{mhchem}

\begin{document}

\subsection{前座}
	\subsubsection{濃度について}
	\subsubsection{標準状態}
標準状態(ひょうじゅんじょうたい)とは、物理学、化学や工学などの分野で、測定する平衡状態に依存する熱力学的な状態量を比較するときに基準とする状態である。標準状態をどのように設定するかは完全に人為的なものであり、理論的な裏付けはないが、歴史的には人間の自然認識に立脚する。 \\
{\large SATP} \\
基準の温度を25セルシウス度(298.15ケルビン)、標準圧力を 100 kPa とするものがSATP(標準環境温度と圧力、英: standard ambient temperature and pressure)と定義される。 \\
{\large STP(1990年頃以降)} \\
基準の温度を0セルシウス度(273.15ケルビン)、標準圧力を 100 kPa とするものがSTP(標準温度と圧力、英: standard temperature and pressure)と定義される。1990年頃より前のSTPはNTPと同じである。 \\
{\large NTP} \\
基準の温度を0セルシウス度(273.15ケルビン)、標準圧力を 101.325 kPa とするものがNTP(標準温度と圧力、英: normal temperature and pressure)と定義される。NTPは1990年頃より前のSTPと同じである。 \\
気体の標準状態としてどの条件が使われるかは、地域や分野により異なる。『アトキンス物理化学要論』によれば2016年現在、主に 25 ℃、1 bar のSATPが使われるが、0 ℃、1 atm のSTPは、今でも使われている。一方『ボール物理化学』によれば、0 ℃、1 bar のSTPが最もふつうの一組である。日本では、単に標準状態といえば 0 ℃、1 atm のNTPを指すことが多い。 \\
1モルの理想気体の体積は、SATPでは24.8リットル、STPでは22.7リットル(1990年頃より前は22.4リットル)、NTPでは22.4リットルである。

	\subsubsection{化学量論}
化学量論とは,化学反応に含まれる単体や化合物との間の定量的な関係をいう. \\
実際の化学反応系において個々の分子の変化をマクロな観測系で観察したり、変化を数え上げることは通常の方法では不可能なので、反応の変化量をマクロ系で測定する為に何らかの測定量(メトリック)が必要となる。一方、化学反応に伴い膨張や収縮など熱力学的状態は変化し、場合によっては相変化も引き起こされることから、物質の測定量としては大きさのような長さの次元を持つものは不適当であり、質量が変化の測定量として採用される。 \\

例えば、気相を無視した観測からは炭素は燃焼(酸化)すると消えて無くなってしまい、化学反応において物質は消滅したりする(或いは逆に無から発生したりする)という誤った結論が導かれてしまう。アントワーヌ・ラヴォアジエの質量保存の法則では測定量として質量を採用し、反応の前後で反応系に出入りする物質が無いように反応系を設定すると、この測定量は保存すると言い表しており、化学反応論の根幹を成す定義である。 \\

この定義に従って反応を観測すると、反応物の変化量と生成物の変化量の間に比例関係があることがわかる。これがプルーストの定比例の法則であり、反応に関与する分子の観点から見ると、複数存在する反応物(場合によっては生成物)の間で、個々の反応毎に関与する量的関係が一意に決まっていることをあらわしており、それらの間には当量関係が存在する。 \\

例に挙げると炭素と酸素から二酸化炭素が発生する反応は、炭素1当量に対して酸素は2当量が反応し、炭素と酸素から一酸化炭素が発生する反応は、炭素1当量に対して酸素は1当量が反応する。これを別の立場で見ると、複数存在する反応物の一方が過剰の場合には、少ないほうの反応物の量で反応可能な量は決まり、余剰な反応物は未反応のまま系内に残存し、それらの量は予測可能であることを意味する。 \\

これらの化学量論の帰結より、化学反応が決まれば反応物の必要量や生成物の期待量を見積もることが可能となる。一方、化学量論による計算結果と実際の測定量の乖離から、反応の収量や元素分析による絶対純度が求められる。このように化学反応の定量関係は化学量論に立脚している。 \\

ある一つの反応物に注目して、その反応物が関与する類似の他の反応で当量関係の間に簡単な整数比が成立するという規則性があり、それがドルトンの倍数比例の法則である(この法則の意味については後述する)。化学反応はドルトンが提言した様に原子(分子)の組み換えであり、それは(化学)反応式で表される。 \\
物質量の概念が化学量論に基づく計算を行うための中核になる。
例えば、100gの酸化亜鉛は80.3gの亜鉛と19.7gの酸素を含む.これを
\[
\ce{80.3gZn} \doteq \ce{19.7gO}
\]
と書くことにする。(略)

\subsubsection{活量}

\subsection{一般化学平衡}
	\subsubsection{反応速度論}
	\subsubsection{質量作用の法則}

\subsection{酸化還元反応}
\subsection{酸塩基反応}
	\subsubsection{連続近似法}
	\subsubsection{緩衝液}
	\subsubsection{ヘンダーソン・ハッセルバルヒの式}
緩衝液のpHを求めるためには、よくヘンダーソン・ハッセルバルヒ式が用いられる。
例として、弱酸\ce{CH3COOH(aq)}とその共役塩基\ce{CH3COO-(aq)}の混合溶液を考える。 \\
このとき、
\[
\ce{CH3COOH(aq) + CH3COO-(aq) <=> CH3COO-(aq) + CH3COOH(aq)}
\]
のようにプロトン移動反応が起こっても、化学種の全濃度には正味の変化は生じない。
これを各化学種の素反応に分けると緩衝液中の平衡は次のようになる。
\begin{align*}
\ce{CH3COOH(aq) + H2O(l) <=> CH3COO-(aq) + H3O+(aq)} \\
K_a = 1.8 \times 10^{-5} M \\
\ce{CH3COO-(aq) + H2O(l) <=> CH3COOH(aq) + OH-(aq)} \\
K_b = 5.6 \times 10^{-10} M
\end{align*}
これらの化学反応に対する平衡定数はいづれも小さいので、どちらの平衡も著しく左側に偏っている。
したがって、溶液中の濃度\ce{[CH3COOH]},\ce{[CH3COO-]}はどちらも各自の化学量論濃度と大きな違いはないものと考えられる。

ゆえに
\begin{align*}
\ce{[CH3COOH] \approx [CH3COOH]0} \\
\ce{[CH3COO-] \approx [CH3COO-]0} \\
\end{align*}
とできる。(右辺はともに化学量論濃度である。) \\
このほかの\ce{H3O+}の供給源としては、水の自己プロトリシス反応がある。 \\
\begin{align*}
\ce{H20(l) + H20(l) <=> H3O+(aq) + OH-(aq)} \\
K_w = 1.0 \times 10^{-14} M^2 \\
\end{align*}
しかし、$K_w$は極めて小さいことからこの寄与を無視すると
次の濃度表を書くことができる。 \\

\begin{center}
\begin{table}[htb]
\begin{tabular}{cccccccc}
濃度&\ce{CH3COOH(aq)}&+&\ce{H2O(l)}&\ce{<=>}&\ce{H3O+(aq)}&+&\ce{CH3COO-(aq)} \\ \hline
初期&0.10 M& & & &$\approx$ 0M& &0.15 M \\
変化量&-x& & & &+x& &+x \\
平衡&0.10 M - x& & & &x& &0.15 M + x
\end{tabular}
\end{table}
\end{center}

平衡濃度の値を平衡定数に代入すると
\[
K_a = \frac{x(0.15M+x)}{0.10M-x} = 1.8 \times 10^{-5} M
\]
$x \ll 0.10M$を考えると、
\[
x \approx 1.2 \times 10^{-5}M
\]
が得られ、近似が妥当であったことがわかる. \\
さて一般に、緩衝液では
\begin{align*}
\ce{[HA] \approx [HA]0} \\
\ce{[A-] \approx [A-]0} \\
\end{align*}
であるから、酸解離反応の平衡定数を用いて、
\begin{align*}
K_a &= \frac{\ce{[H+][A-]}}{\ce{[HA]}} \\
&\approx \frac{\ce{[H+][A-]0}}{\ce{[HA]0}}
\end{align*}
\[
\ce{[H+] \approx \frac{K_a[HA]0}{[A-]0}}
\]
ゆえに
\[
pH \approx pK_a + log \frac{\ce{[Base]0}}{\ce{[Acid]0}}
\]


\subsection{気体反応}
\subsection{浸透圧}
\subsection{束一性}
\subsection{}
\subsection{}
\subsection{}
\subsection{}
\subsubsection{}


\end{document}
