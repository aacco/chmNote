\documentclass[dvipdfmx,uplatex]{jsarticle}
\usepackage{macros}
\usepackage[version=3]{mhchem}
\usepackage{url}

\begin{document}

\subsection{総論}
\subsubsection{濃度について}
\subsubsection{標準状態}
標準状態(ひょうじゅんじょうたい)とは、物理学、化学や工学などの分野で、測定する平衡状態に依存する熱力学的な状態量を比較するときに基準とする状態である。標準状態をどのように設定するかは完全に人為的なものであり、理論的な裏付けはないが、歴史的には人間の自然認識に立脚する。 \\
{\Large SATP} \\
基準の温度を25セルシウス度(298.15ケルビン)、標準圧力を 100 kPa とするものがSATP(標準環境温度と圧力、英: standard ambient temperature and pressure)と定義される。 \\
{\Large STP(1990年頃以降)} \\
基準の温度を0セルシウス度(273.15ケルビン)、標準圧力を 100 kPa とするものがSTP(標準温度と圧力、英: standard temperature and pressure)と定義される。1990年頃より前のSTPはNTPと同じである。 \\
{\Large NTP} \\
基準の温度を0セルシウス度(273.15ケルビン)、標準圧力を 101.325 kPa とするものがNTP(標準温度と圧力、英: normal temperature and pressure)と定義される。NTPは1990年頃より前のSTPと同じである。 \\
1モルの理想気体の体積は、SATPでは24.8リットル、STPでは22.7リットル(1990年頃より前は22.4リットル)、NTPでは22.4リットルである。

	\subsubsection{化学量論}
化学量論とは,化学反応に含まれる単体や化合物との間の定量的な関係をいう. \\
物質量の概念が化学量論に基づく計算を行うための中核になる。
例えば、100gの酸化亜鉛は80.3gの亜鉛と19.7gの酸素を含む.これを
\[
\ce{80.3gZn} \doteq \ce{19.7gO}
\]
と書くことにする。(略)

\subsubsection{活量}

\subsubsection{収支}
保存則に基づいて収支の式を立てることが重要なことがある。
収支は \\
\begin{itemize}
\item 量論収支
\item 電気的収支
\end{itemize}
があり、物質量、化学量論や質量保存則に基づいて立てられる。

\subsubsection{化学反応}
化学反応は電子の移動にともなって結合の切断と生成が行われる。化学結合と電子の移動方法に着目して化学反応を分類すると、イオン反応 (ionic reaction)、ラジカル反応 (free-radical reaction)、ペリ環状反応 (pericyclic reaction) に大別される。ある化学種がもうひとつの化学種と結合をつくる反応について考えると、イオン反応は、一方の化学種から電子対が供与されて新しい結合性軌道が生成する化学反応で、電子求引性や電子供与性など原子間の電荷の偏りにより反応の方向が支配される。ラジカル反応は、双方の化学種から1電子ずつ電子が供与されて新しい結合性軌道が生成する化学反応である。ペリ環状反応は、化学種のπ軌道からσ軌道へ、環状の遷移状態を経て転化することで2ヶ所以上に新たな結合が生成する化学反応である。切断は結合の逆反応にあたる。 \\
反応機構や、反応物と生成物の構成の違いで化学反応を考える場合、置換反応、付加反応、脱離反応、転位反応などに分類される。 \\
加水分解、脱水反応、付加重合、縮合重合(縮重合)、酸化反応、還元反応、中和反応は化学反応の用途を意識した分類で、上記 4 反応機構の一つあるいは複数から構成される。 \\
\url{https://ja.m.wikipedia.org/wiki/化学反応} \\
化学反応は基本的に電子論で説明され、フロンティア電子理論などがある。


\subsection{一般化学平衡}
\subsubsection{反応速度論}
\subsubsection{質量作用の法則}
平衡状態の化学ポテンシャルのつり合いから。

\subsection{酸化還元反応}
電子論で説明できる。一般に電気陰性度が多きい非金属元素単体は酸化剤になり

\subsection{酸塩基反応}
\subsubsection{連続近似法}
\url{https://en.m.wikipedia.org/wiki/Methods_of_successive_approximation}

\subsubsection{緩衝液}
\subsubsection{ヘンダーソン・ハッセルバルヒの式}
\begin{theo}[ヘンダーソン・ハッセルバルヒの式] \mbox{} \\
酸性緩衝液において、酸解離定数を$K_a$とすると \\
\[
pH \approx pK_a + log \frac{\ce{[Base]0}}{\ce{[Acid]0}}
\]
(塩基の場合も同様)
\end{theo}

緩衝液のpHを求めるためには、よくヘンダーソン・ハッセルバルヒ式が用いられる。
例として、弱酸\ce{CH3COOH(aq)}とその共役塩基\ce{CH3COO- (aq)}の混合溶液を考える。 \\
このとき、
\[
\ce{CH3COOH(aq) + CH3COO- (aq) <=> CH3COO- (aq) + CH3COOH(aq)}
\]
のようにプロトン移動反応が起こっても、化学種の全濃度には正味の変化は生じない。
これを各化学種の素反応に分けると緩衝液中の平衡は次のようになる。
\begin{align*}
\ce{CH3COOH(aq) + H2O(l) <=> CH3COO- (aq) + H3O+(aq)} \\
K_a = 1.8 \times 10^{-5} M \\
\ce{CH3COO- (aq) + H2O(l) <=> CH3COOH(aq) + OH- (aq)} \\
K_b = 5.6 \times 10^{-10} M
\end{align*}
これらの化学反応に対する平衡定数はいづれも小さいので、どちらの平衡も著しく左側に偏っている。
したがって、溶液中の濃度\ce{[CH3COOH]},\ce{[CH3COO- ]}はどちらも各自の化学量論濃度と大きな違いはないものと考えられる。

ゆえに
\begin{align*}
\ce{[CH3COOH] \approx [CH3COOH]0} \\
\ce{[CH3COO- ] \approx [CH3COO- ]0} \\
\end{align*}
とできる。(右辺はともに化学量論濃度である。) \\
このほかの\ce{H3O+}の供給源としては、水の自己プロトリシス反応がある。 \\
\begin{align*}
\ce{H20(l) + H20(l) <=> H3O+(aq) + OH- (aq)} \\
K_w = 1.0 \times 10^{-14} M^2 \\
\end{align*}
しかし、$K_w$は極めて小さいことからこの寄与を無視すると
次の濃度表を書くことができる。 \\

\begin{center}
\begin{table}[htb]
\begin{tabular}{cccccccc}
濃度&\ce{CH3COOH(aq)}&+&\ce{H2O(l)}&\ce{<=>}&\ce{H3O+(aq)}&+&\ce{CH3COO- (aq)} \\ \hline
初期&0.10 M& & & &$\approx$ 0M& &0.15 M \\
変化量&-x& & & &+x& &+x \\
平衡&0.10 M - x& & & &x& &0.15 M + x
\end{tabular}
\end{table}
\end{center}

平衡濃度の値を平衡定数に代入すると
\[
K_a = \frac{x(0.15M+x)}{0.10M-x} = 1.8 \times 10^{-5} M
\]
$x \ll 0.10M$を考えると、
\[
x \approx 1.2 \times 10^{-5}M
\]
が得られ、近似が妥当であったことがわかる. \\
さて一般に、緩衝液では
\begin{align*}
\ce{[HA] \approx [HA]0} \\
\ce{[A- ] \approx [A- ]0} \\
\end{align*}
であるから、酸解離反応の平衡定数を用いて、
\begin{align*}
K_a &= \frac{\ce{[H+][A- ]}}{\ce{[HA]}} \\
&\approx \frac{\ce{[H+][A- ]0}}{\ce{[HA]0}}
\end{align*}
\[
\ce{[H+] \approx \frac{K_a[HA]0}{[A- ]0}}
\]
ゆえに
\[
pH \approx pK_a + log \frac{\ce{[Base]0}}{\ce{[Acid]0}}
\]


\subsection{気体反応}
相不変の気体と気液平衡の気体があるとき、まずすべてが気相であると仮定した上での圧力と蒸気圧の大小で相を決定する。 \\
すべて気相になるとき、気液平衡を仮定すると状態図との矛盾または必要なモル比を満たさないことがわかる。

\subsection{浸透圧}
\begin{defi}[浸透圧] \mbox{} \\
半透膜を挟んで液面の高さが同じ、溶媒のみの純溶媒と溶液がある時、純溶媒から溶液へ溶媒が浸透するが、溶液側に圧を加えると浸透が阻止される。この圧を溶液の浸透圧という。
\end{defi}
\begin{theo}[van't Hoffの式] \mbox{} \\
あまり濃厚でない溶液について次が成り立つ。 \\
\[
\Pi = iCRT
\]
($\Pi$は浸透圧、iはファント・ホッフ定数、Cは体積モル濃度)
\end{theo}
電解質のとき$i>1$で、ファント・ホッフ定数から電離度が推定できる。

\subsection{束一性}
\subsection{}
\subsection{}
\subsection{}
\subsection{}
\subsubsection{}


\end{document}
