\documentclass[dvipdfmx,uplatex]{jsarticle}
\usepackage{macros}

\begin{document}
\subsection{abio}

%\begin{table}[]
%	\begin{tabular}[]
%
%	\end{tabular}
%\end{table}

\subsubsection{}
\begin{align*}
M_mO_n + H_2O \to M(OH)_n
\end{align*}
or \\
\begin{align*}
M_mO_n + H_2O \to MO_k(OH)_l
\end{align*}
e.g. 
\begin{align*}
CO_2 + H_2O	&\to 	H_2CO_3 \ i.e. \ {CO(OH)_2} \\
SiO_2 + H_2O	&\to	H_2SiO_3 \ i.e. \ {SiO(OH)_2} \\
N_2O_5 + H_2O	&\to	2HNO_3\ i.e. \ {NO_2(OH)} \\
P_4O_{10} + 6H_2O	&\to	4H_3PO_4\ i.e. \ {PO(OH)_3} \\
SO_2 + H_2O	&\to	H_2SO_3\ i.e. \ SO(OH)_2 \\
SO_3 + H_2O	&\to	H_2SO_4\ i.e. \ SO_2(OH)_2 \\
Cl_2O_7	+ H_2O	&\to	HClO_4\ i.e. \ {ClO_3OH} \\
\end{align*}
酸化物水溶液のpHは電気陰性度による.ゆえに金属元素の酸化物は塩基性になりやすく,非金属元素のそれは酸性になりやすい.
\end{document}
