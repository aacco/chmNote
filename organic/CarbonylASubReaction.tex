\documentclass[dvipdfmx,uplatex]{jsarticle}
\usepackage{../sty/macros}

\begin{document}
\subsection{ハロホルム反応}
\begin{enumerate}
	\item ルイス塩基によりカルボニル基に隣接する$\alpha$位のメチル基からプロトン($\alpha$水素)が引き抜かれエノラートイオンが生成する。
	\item 生成したエノラートがハロゲン化剤に対して求核攻撃してカルボニル基の$\alpha$位がハロゲン原子に置換される。
	\item ハロゲン原子への置換が起こるとその炭素上の水素の酸性度が上がるため、よりエノラートができやすくなりこの過程が繰り返される。結果としてメチル基のすべての水素がハロゲンに置換され,トリハロメチルケトンが生成する.
	\item 水酸化物イオンがカルボニル基に求核攻撃することで付加脱離反応が進行し、トリハロメチルアニオンと、カルボン酸が生じる。
	\item 溶媒が水の場合は、水からトリハロメチルアニオンがプロトンを引き抜いてトリハロメタンになる。ここで生じた水酸化物イオンにより、塩基性条件が保たれるのである。
\end{enumerate}


\end{document}
